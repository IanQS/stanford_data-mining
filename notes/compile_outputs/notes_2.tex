%%%%%%%%%%%%%%%%%%%%%%%%%%%%%%%%%%%%%%%%%
% Structured General Purpose Assignment
% LaTeX Template
%
% This template has been downloaded from:
% http://www.latextemplates.com
%
% Original author:
% Ted Pavlic (http://www.tedpavlic.com)
%
% Note:
% The \lipsum[#] commands throughout this template generate dummy text
% to fill the template out. These commands should all be removed when
% writing assignment content.
%
%%%%%%%%%%%%%%%%%%%%%%%%%%%%%%%%%%%%%%%%%

%----------------------------------------------------------------------------------------
%	PACKAGES AND OTHER DOCUMENT CONFIGURATIONS
%----------------------------------------------------------------------------------------

\documentclass{article}

\usepackage{fancyhdr} % Required for custom headers
\usepackage{lastpage} % Required to determine the last page for the footer
\usepackage{extramarks} % Required for headers and footers
\usepackage{graphicx} % Required to insert images
\usepackage{lipsum} % Used for inserting dummy 'Lorem ipsum' text into the
% template
\usepackage{amsmath}

% Margins
\topmargin=-0.45in
\evensidemargin=0in
\oddsidemargin=0in
\textwidth=6.5in
\textheight=9.0in
\headsep=0.25in

\linespread{1.1} % Line spacing

% Set up the header and footer
\pagestyle{fancy}
\lhead{\hmwkAuthorName} % Top left header
\chead{\hmwkClass\ (\hmwkTitle)} % Top center header
\rhead{\firstxmark} % Top right header
\lfoot{\lastxmark} % Bottom left footer
\cfoot{} % Bottom center footer
\rfoot{Page\ \thepage\ of\ \pageref{LastPage}} % Bottom right footer
\renewcommand\headrulewidth{0.4pt} % Size of the header rule
\renewcommand\footrulewidth{0.4pt} % Size of the footer rule

\setlength\parindent{0pt} % Removes all indentation from paragraphs

%----------------------------------------------------------------------------------------
%	DOCUMENT STRUCTURE COMMANDS
%	Skip this unless you know what you're doing
%----------------------------------------------------------------------------------------

% Header and footer for when a page split occurs within a problem environment
\newcommand{\enterProblemHeader}[1]{
\nobreak\extramarks{#1}{#1 continued on next page\ldots}\nobreak
\nobreak\extramarks{#1 (continued)}{#1 continued on next page\ldots}\nobreak
}

% Header and footer for when a page split occurs between problem environments
\newcommand{\exitProblemHeader}[1]{
\nobreak\extramarks{#1 (continued)}{#1 continued on next page\ldots}\nobreak
\nobreak\extramarks{#1}{}\nobreak
}

\setcounter{secnumdepth}{0} % Removes default section numbers
\newcounter{homeworkProblemCounter} % Creates a counter to keep track of the number of problems

\newcommand{\homeworkProblemName}{}
\newenvironment{homeworkProblem}[1][Topic \arabic{homeworkProblemCounter}]{ % Makes a new environment called homeworkProblem which takes 1 argument (custom name) but the default is "Problem #"
\stepcounter{homeworkProblemCounter} % Increase counter for number of problems
\renewcommand{\homeworkProblemName}{#1} % Assign \homeworkProblemName the name of the problem
\section{\homeworkProblemName} % Make a section in the document with the custom problem count
\enterProblemHeader{\homeworkProblemName} % Header and footer within the environment
}{
\exitProblemHeader{\homeworkProblemName} % Header and footer after the environment
}

\newcommand{\problemAnswer}[1]{ % Defines the problem answer command with the content as the only argument
\noindent\framebox[\columnwidth][c]{\begin{minipage}{0.98\columnwidth}#1\end{minipage}} % Makes the box around the problem answer and puts the content inside
}

\newcommand{\homeworkSectionName}{}
\newenvironment{homeworkSection}[1]{ % New environment for sections within homework problems, takes 1 argument - the name of the section
\renewcommand{\homeworkSectionName}{#1} % Assign \homeworkSectionName to the name of the section from the environment argument
\subsection{\homeworkSectionName} % Make a subsection with the custom name of the subsection
\enterProblemHeader{\homeworkProblemName\ [\homeworkSectionName]} % Header and footer within the environment
}{
\enterProblemHeader{\homeworkProblemName} % Header and footer after the environment
}

%----------------------------------------------------------------------------------------
%	NAME AND CLASS SECTION
%----------------------------------------------------------------------------------------

\newcommand{\hmwkClass}{Mining Massive Datasets} % Course/class
\newcommand{\hmwkTitle}{Week 2: PageRank} % Assignment title
\newcommand{\hmwkClassTime}{-} % Class/lecture time
\newcommand{\hmwkAuthorName}{Ian Quah (itq)} % Your name

%----------------------------------------------------------------------------------------
%	TITLE PAGE
%----------------------------------------------------------------------------------------

\title{
\vspace{2in}
\textmd{\textbf{\hmwkClass:}\\
\textmd{\hmwkTitle}}\\
\vspace{3in}
}

\author{\textbf{\hmwkAuthorName}}

%----------------------------------------------------------------------------------------

\begin{document}

\maketitle

%----------------------------------------------------------------------------------------
%	PROBLEM 1
%----------------------------------------------------------------------------------------

% To have just one problem per page, simply put a \clearpage after each problem

\newpage
\begin{homeworkProblem}{\textbf{Link Analysis and PageRank}}

  \problemAnswer{
  Web as a graph:
  \begin{enumerate}
  \item nodes: webpage
  \item edge: hyperlink
  \end{enumerate}

  Link Analysis Approach:
  \begin{enumerate}
    \item Page Rank - Google
    \item Hubs and Authorities
    \item Topic-Specific Page Rank
    \item Web Spam Detection Algorithms
  \end{enumerate}
}
\end{homeworkProblem}

\begin{homeworkProblem}{\textbf{PageRank: Flow Formulation}}

  \problemAnswer{
    \begin{enumerate}
    \item     Let r$_j$ be the rank of website j (normalized to 1)
    \item     Flow Algorithm: r$_j$ = $\sum_{i \rightarrow j} \frac{r_i}{|D|}$, where $|D|$ is num of
      nodes pointing to j
    \item Vote from important page is worth more
    \item  However, when there are a lot of pages, this computation becomes expensive, so
  we need to reformulate it.
    \end{enumerate}
}
\end{homeworkProblem}


\begin{homeworkProblem}{\textbf{PageRank: Matrix}}

  \problemAnswer{
    \begin{enumerate}
    \item r$_j$ = $\sum_{i \rightarrow j} \frac{r_i}{|D|}$ $\rightarrow$ r = M $\cdot$ r
    \item If we constrain the sum of the outnodes to 1, then the matrix is
      called Column Stochastic
    \item Largest Eigenvalue of M is 1, since M is column stochastic

      A x = $\lambda $x

    \item We can now efficiently solve for r using Power Iteration
    \end{enumerate}
}
\end{homeworkProblem}


\begin{homeworkProblem}{\textbf{PageRank: Power Iteration}}

  \problemAnswer{

    Power Iteration: iterative scheme

    \begin{enumerate}
    \item     Given web graph with N nodes.
    \begin{enumerate}
    \item N web pages
    \item \textbf{Initialize}: r$^{(0)}$ = [1/N, ...., 1/N]$^T$
    \item \textbf{Iterate}: r$^{(t+1)}$ = M $\cdot$ r$^t$
    \item \textbf{Halting condition}: $|$ r$^{(t+1)}$ - r$^t$ $|$ $<$ $\epsilon$, here
      using L$_1$ norm but can use any other norm
    \end{enumerate}
    \item Intuition: keep multiplying it until it converges (reaches stationary state)
    \end{enumerate}


    Formulation as a random walk
    \begin{enumerate}
    \item At time t, random walker is on page i
    \item At t + 1, walker randomly picks a page out (out-link)
    \item Ends up on some page j from i (M$_{ji}$)
    \item Repeats indefinitely
    \end{enumerate}

    Since the out-links are normalized, this is a probability distribution

}
\end{homeworkProblem}

\begin{homeworkProblem}{\textbf{Google formulation of Page Rank}}

  \vspace{0.5 cm}

  \textbf{Page rank: 3 Questions}
  \begin{enumerate}
  \item Does the result converge? \hfill Not necessarily
  \item Does it converge to what we want? \hfill Not always
  \item Are the results reasonable? \hfill Not always
  \end{enumerate}

  \textbf{Google's Solutions}
  \begin{enumerate}
  \item Spider Traps: random surfers teleporting out with probability $\beta$
  \item Dead Ends: random surfer teleports out with probability 1, to another
    node (uniformly)
  \end{enumerate}

  \textbf{Why do random surfers solve our problem?}

  \begin{enumerate}
  \item It makes our transition matrix, M, stochastic

    Because what ends up happening is that because there is some chance that it
    can teleport to any page (uniformly), it becomes a dense graph, and because
    it can do it with equal probability, this does not actually change the
    probabilities beyond just making sure that our matrix in the limit does not
    become 0 or 1 (caught in spider trap)

  \item Makes M aperiodic
    \begin{enumerate}
    \item \textbf{Def}
      A chain is aperiodic if there exists k $>$ 1 s.t interval between two
      visits to some state S is always a multiple of k
    \end{enumerate}
  \item Makes M irreducible
    \begin{enumerate}
    \item \textbf{Def}
      Non-0 probability of transitioning out of any node
    \end{enumerate}
  \end{enumerate}

  \textbf{The Google Matrix}

  \problemAnswer{

  \textbf{New form of our PageRank equation}

  \[r_j = \Sigma_{i \rightarrow j} \beta\frac{r_i}{d_i} + (1 - \beta)\frac{1}{n} \]

  \textbf{Google Matrix A, modifying M}

  \[ A = \beta M + (1 - \beta)\frac{1}{n}\emph{e}\cdot\emph{e}\] \hfill where e is vector of
  1's

  We know that A is stochastic, aperiodic and irreducible some

  \[r^{t + 1} = A \cdot r^{(t)}\]

  in practice, $\beta$ is typically 0.8 or 0.9, so our surfer makes about 5 steps
  and then jumps

}
\end{homeworkProblem}

\begin{homeworkProblem}{\textbf{Computing PageRank for web-scale}}

  \textbf{Matrix A vs M}

  We would prefer to work with matrix M rather than matrix A just because M is
  full of 0's which is cheap to store, rather than matrix A which is dense

  \textbf{Investigating the matrix}

  \problemAnswer{
    \begin{align*}
      \text{Recall: r} &= A\cdot r \text{ where } A_{ij} = \beta M_{ij} + \frac{1 - \beta}{N}\\
      r_i &= \Sigma_{j=1}^N A_{ij} \cdot r_j\\
      r_i &= \Sigma_{j=1}^N [\beta M_{ij} + \frac{1 - \beta}{N}] \cdot r_j \tag{Substitute in A}\\
      r_i &= \Sigma_{j=1}^N \beta M_{ij}  r_j + \Sigma_{j=1}^N \frac{1 - \beta}{N} \cdot r_j \tag{Distribute the sums}\\
      &= \Sigma_{j=1}^N \beta M_{ij}  r_j +\frac{1 - \beta}{N} \tag{bc sum of r$_j$ = 1}\\
      r &= \beta M \cdot r + [\frac{1 - \beta}{N}]_N \\
    \end{align*}
    $[\frac{1 - \beta}{N}]_N $ is a vector of length N, where all elems have value $\frac{1 - \beta}{N}$
  }

  \textbf{Full Page Rank Algorithm}

  \includegraphics[width=13cm]{e_1}


\end{homeworkProblem}


\begin{homeworkProblem}{\textbf{Assignment 2 hints}}

  My personal hints are to just use the matrix form of the equation rather than
  implementing the iterative method. Both lead to the same answer, and the
  matrix form is much cleaner.

\end{homeworkProblem}

\end{document}
