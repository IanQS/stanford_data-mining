%%%%%%%%%%%%%%%%%%%%%%%%%%%%%%%%%%%%%%%%%
% Structured General Purpose Assignment
% LaTeX Template
%
% This template has been downloaded from:
% http://www.latextemplates.com
%
% Original author:
% Ted Pavlic (http://www.tedpavlic.com)
%
% Note:
% The \lipsum[#] commands throughout this template generate dummy text
% to fill the template out. These commands should all be removed when
% writing assignment content.
%
%%%%%%%%%%%%%%%%%%%%%%%%%%%%%%%%%%%%%%%%%

%----------------------------------------------------------------------------------------
%	PACKAGES AND OTHER DOCUMENT CONFIGURATIONS
%----------------------------------------------------------------------------------------

\documentclass{article}
\usepackage{amsfonts}
\usepackage{fancyhdr} % Required for custom headers
\usepackage{lastpage} % Required to determine the last page for the footer
\usepackage{extramarks} % Required for headers and footers
\usepackage{graphicx} % Required to insert images
\usepackage{lipsum} % Used for inserting dummy 'Lorem ipsum' text into the
% template
\usepackage{amsmath}

% Margins
\topmargin=-0.45in
\evensidemargin=0in
\oddsidemargin=0in
\textwidth=6.5in
\textheight=9.0in
\headsep=0.25in

\linespread{1.1} % Line spacing

% Set up the header and footer
\pagestyle{fancy}
\lhead{\hmwkAuthorName} % Top left header
\chead{\hmwkClass\ (\hmwkTitle)} % Top center header
\rhead{\firstxmark} % Top right header
\lfoot{\lastxmark} % Bottom left footer
\cfoot{} % Bottom center footer
\rfoot{Page\ \thepage\ of\ \pageref{LastPage}} % Bottom right footer
\renewcommand\headrulewidth{0.4pt} % Size of the header rule
\renewcommand\footrulewidth{0.4pt} % Size of the footer rule

\setlength\parindent{0pt} % Removes all indentation from paragraphs

%----------------------------------------------------------------------------------------
%	DOCUMENT STRUCTURE COMMANDS
%	Skip this unless you know what you're doing
%----------------------------------------------------------------------------------------

% Header and footer for when a page split occurs within a problem environment
\newcommand{\enterProblemHeader}[1]{
\nobreak\extramarks{#1}{#1 continued on next page\ldots}\nobreak
\nobreak\extramarks{#1 (continued)}{#1 continued on next page\ldots}\nobreak
}

% Header and footer for when a page split occurs between problem environments
\newcommand{\exitProblemHeader}[1]{
\nobreak\extramarks{#1 (continued)}{#1 continued on next page\ldots}\nobreak
\nobreak\extramarks{#1}{}\nobreak
}

\setcounter{secnumdepth}{0} % Removes default section numbers
\newcounter{homeworkProblemCounter} % Creates a counter to keep track of the number of problems

\newcommand{\homeworkProblemName}{}
\newenvironment{homeworkProblem}[1][Topic \arabic{homeworkProblemCounter}]{ % Makes a new environment called homeworkProblem which takes 1 argument (custom name) but the default is "Problem #"
\stepcounter{homeworkProblemCounter} % Increase counter for number of problems
\renewcommand{\homeworkProblemName}{#1} % Assign \homeworkProblemName the name of the problem
\section{\homeworkProblemName} % Make a section in the document with the custom problem count
\enterProblemHeader{\homeworkProblemName} % Header and footer within the environment
}{
\exitProblemHeader{\homeworkProblemName} % Header and footer after the environment
}

\newcommand{\problemAnswer}[1]{ % Defines the problem answer command with the content as the only argument
\noindent\framebox[\columnwidth][c]{\begin{minipage}{0.98\columnwidth}#1\end{minipage}} % Makes the box around the problem answer and puts the content inside
}

\newcommand{\homeworkSectionName}{}
\newenvironment{homeworkSection}[1]{ % New environment for sections within homework problems, takes 1 argument - the name of the section
\renewcommand{\homeworkSectionName}{#1} % Assign \homeworkSectionName to the name of the section from the environment argument
\subsection{\homeworkSectionName} % Make a subsection with the custom name of the subsection
\enterProblemHeader{\homeworkProblemName\ [\homeworkSectionName]} % Header and footer within the environment
}{
\enterProblemHeader{\homeworkProblemName} % Header and footer after the environment
}

%----------------------------------------------------------------------------------------
%	NAME AND CLASS SECTION
%----------------------------------------------------------------------------------------

\newcommand{\hmwkClass}{Mining Massive Datasets} % Course/class
\newcommand{\hmwkTitle}{Week 4: Distance Measures} % Assignment title
\newcommand{\hmwkClassTime}{-} % Class/lecture time
\newcommand{\hmwkAuthorName}{Ian Quah (itq)} % Your name

%----------------------------------------------------------------------------------------
%	TITLE PAGE
%----------------------------------------------------------------------------------------

\title{
\vspace{2in}
\textmd{\textbf{\hmwkClass:}\\
\textmd{\hmwkTitle}}\\
\vspace{3in}
}

\author{\textbf{\hmwkAuthorName}}

%----------------------------------------------------------------------------------------

\begin{document}

\maketitle

%----------------------------------------------------------------------------------------
%	PROBLEM 1
%----------------------------------------------------------------------------------------

% To have just one problem per page, simply put a \clearpage after each problem

{\LARGE Note: The notes here are sparse because I know most of the material
  discussed so definitely go through it on your own}

\newpage
\begin{homeworkProblem}{\Large \textbf{Distance Measures}}

  \textbf{Distances Measures}
  \begin{enumerate}
  \item Generalized LSH is based on notion of distance between points
  \item Note: Jaccard Similarity is not a true distance, 1 - Jaccard is

  \item \textbf{Axioms of distance functions}

    D is distance function on x,y: D(x,y) if

    \begin{enumerate}

    \item D(x, y) $\geq$ 0
    \item D(x, y) = 0 iff x == y
    \item D(x,y) = d(y, x)
    \item d(x, y) $\leq$ d(x, z) + d(z, y) \hfill The triangle inequality
    \end{enumerate}

  \item \textbf{Euclidean}

    \begin{enumerate}
    \item has some number of real-valued dimensions and dense points
    \item There is a notion of ``average'' of two points - useful for thinking
      about clusters
    \item E.g: L$_1$, L$_2$, L$_{\infty}$
    \end{enumerate}

  \item \textbf{Non-Euclidean}
    \begin{enumerate}
    \item Based on properties of points, not location in a space
    \item If distance measure is not Euclidean, automatically non-Euclidean
    \item E.g: Jaccard, Cosine, Edit, Hamming Distance
    \end{enumerate}
  \end{enumerate}
\end{homeworkProblem}

\begin{homeworkProblem}{\Large \textbf{Nearest Neighbor Learning}}

  \textbf{Instance based learning}
  \begin{enumerate}
  \item Run classification again for each new example (unlike other algorithms
    where we estimate some parameters $\theta$ which we use to speed up
    classification on new params)
  \item \textbf{K-nearest Neighbors}
    \begin{enumerate}
    \item Works for regression and classification
    \item Keep whole training dataset
    \item New query, q comes in
    \item Find closest examples X
    \item Predict y$_q$
    \end{enumerate}

  \item Real world example: Collaborative filtering

  \item \textbf{KNN for large datasets}
    \begin{enumerate}
    \item \textbf{Given}: set of point P, s.t each point $\in$ $\mathbb{R}^d$
    \item \textbf{Goal:} Given a query point q
    \item \textbf{NN}: Find nearest neighbor p of q in P
    \item \textbf{Range search} Find one/ all points in P within distance r from
      q

    \item Two types of queries when dealing with NN

      1) Find K nearest to query point q

      2) Find all points within some distance r to q

      O(n), but with locality sensitive hashing, can be near constant
    \end{enumerate}

  \end{enumerate}

\end{homeworkProblem}

\end{document}
